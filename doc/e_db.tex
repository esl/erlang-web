\subsection{DBMS}
\subsubsection{Overview}
During the build of your application we will notice there is a very narrow set of most common database operations:
\begin{itemize}
\item saving the element
\item reading the element with the given ID
\item reading all the elements from the given domain
\item updating the existing element
\item deleting the element with the given ID
\item obtaining the next available ID within the given domain
\end{itemize}

The {\it e\_db} module provides the convenient API to all those operations.

\subsubsection{e\_db module}
{\it e\_db} module is a transparent API to the DBMS laying at the bottom of the system. It allows us to run the following operations on the database:
\begin{description}
\item[install()]- installs the selected database. When we are using mnesia it creates the table for used ID's. Otherwise, when we are using CouchDB, creates two new databases: one for our project (with our project name - it is specified in {\it project.conf} file) and one for our project's ID's.
\item[write(Domain, Element)]- saves the {\it Element} to the database in the specified {\it Domain}.
\item[read(Domain)]- reads and returns a list of all the entities from the given {\it Domain}.
\item[read(Domain, Id)]- reads the element from the {\it Domain} with the given {\it Id}.
\item[update(Domain, Element)]- updates the {\it Element} in the given {\it Domain}.
\item[size(Domain)]- returns the number of elements stored in the given {\it Domain}.
\item[delete(Domain, Element)]- deletes {\it Element} from the {\it Domain}.
\item[get\_next\_id(Domain)]- returnes the next, unique ID within the {\it Domain}.
\end{description}

\subsubsection{Supported DBMS}
Currently only {\it Mnesia} and {\it CouchDB} DBMS are supported. Moreover, {\it CouchDB} support is in incubation phase, so it is not recommended to use it in the production systems. To set the desired DBMS we should specify its name in the {\it project.conf} file:
\begin{verbatim}
{dbms, DBMS}.
\end{verbatim}
where {\bf DBMS} is either {\it mnesia} for Mnesia support or {\it couchdb} for CouchDB. 

After setting this option, the access to the selected DBMS is completely transparent when we are using {\it e\_db} module.

\subsubsection{Example}
Let's check out the basic use cases of {\it e\_db} module:
\begin{verbatim}
%% creating new item
ID = e_db:get_next_id(blog_post),
Post = blog_post:create_post(ID),
e_db:write(blog_post, Post),
...
%% reading all the items
Posts = e_db:read(blog_post),
...
%% reading the particular item
Post = e_db:read(blog_post, 10),
...
%% updating the existing item
Post = e_db:read(blog_post, 3),
NewPost = blog_post:update_post(Post),
e_db:update(blog_post, NewPost),
...
%% removing the item
Post = e_db:read(blog_post, 5),
e_db:delete(blog_post, Post),
...
%% getting the domain size
Size = e_db:size(blog_post),
...
\end{verbatim}
