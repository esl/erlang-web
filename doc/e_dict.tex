\subsection{Request Dictionary}
\subsubsection{Overview}
Each request to HTTP server mod sets up \textit{request dictionary}. Its life time
is limited only to this one request processing time and after that it is cleaned up. 
Module which provides an API to it is {\it wparts}, however it was introduce
only to improve handling of dictionary values. It is developers decision to
brake framework idea of calling \emph{request dictionary} through \emph{wpart}
application and work directly with \emph{eptic} --- it can give some
advantages in specific situations. 
\nlparagraph{Usage}
In general, it is possible to set 
\{Key, Value\} and later get those values from dictionary (they will be stored under 
the {\it Key}). Below is presented a list of functions operating on request
dictionary. 

\subsubsection{Request dictionary API}
\nlparagraph{Wpart API}
\emph{Dictionary} API covers functions:\emph{ fset/2, finsert/2,
  fget/1}. Altohough \emph{fset} takes only two arguments \{ \emph{Key},
\emph{Value} \}  it is possible to create deeper structures. In verion 1.1 it is
  only 2-level tree. To separate parent node from leaves in \emph{Key}
  \emph{":"} (colon) is used. \emph{Wpart} API uses only string as keys,
  thanks to that it is possible to call \emph{dictionary} from \emph{View}
  application level (\emph{wpart:lookup} look for spec in Manual).  \\ \\
How to put values into dictionary?
\begin{description}
  \item[fset/2] 
\begin{Verbatim}
      wpart:fset("SomeKey", SomeVal),
\end{Verbatim}
      All different values stored under \emph{Key} are in dictionary during request time.

  \item[fset/2] - possible to create family of values in one table \begin{verbatim}
      wpart:fset("List:Key",Val),
    \end{verbatim}
  \item[finsert/2] - analogical structure like fset but replaces values if the
    key is the same \begin{verbatim}
      wpart:finsert("key", Val),
    \end{verbatim}  
\end{description}
\clearpage
How to get values from dict?
\begin{description}
  \item[fget/1] \begin{verbatim}
      wpart:fset("SomeKey"),
    \end{verbatim}
  \item [fget/1]  \begin{verbatim}
      wpart:fget("People:Developers"),
    \end{verbatim}
    If there is more values under one key function returns a list of them.
\end{description}
\nlparagraph{Low level calls API}
Setting values
\begin{description}
  \item[fset/2] 
\begin{Verbatim}
      eptic:fset("some key", SomeVal),
\end{Verbatim}
   
    Even if some values are set under one key they are not missing. {\large\textsc{This is
    major change from version 1.0.}} It has been implemented because HTML has some
    tag constructions which do not allow to set name parameters in dependent
    tags - just the cover tag has it.
  \item[fset/3] - possible to create family of values in one table \begin{verbatim}
      eptic:fset("list","key",Val),
    \end{verbatim}
  \item[finsert/2] - analogical structure like fset but replaces values if the
    key is the same \begin{verbatim}
      eptic:finsert("key", Val),
    \end{verbatim}  
\end{description}
Recovering values
\begin{description}
  \item[fget/1] \begin{verbatim}
      eptic:fset("some key"),
    \end{verbatim}
  \item [fget/2]  \begin{verbatim}
      eptic:fget("post","name1"),
    \end{verbatim}
    If there is more values under one key function returns a list of them
\end{description}
\clearpage
\subsubsection{Special cases}
Dictionary is also used by framework. E.g. POST messages from browser go over
it. Each message on framework side passed by request dictionary should start with double
underscore.
Because of backward compatibility issues some of them still are not. Below is the list of
restricted keys. \newline \newline
\begin{tabular}{|l|l|}
  \hline
  {\bf key} & {\bf description}\\
  \hline
  \_\_https &  Bool() \\
  \_\_controller & current controller\\
  post & POST\\
  get & GET\\
  \_\_not\_validated & record from form which failed validation\\
  \_\_error & reason of validation failure\\
  \_\_type & used by wpart\_page - information about whelp \\ 
  & which  feeds wpart\_page list\\
  \_\_types & dynamic feed for types addressed to wpart\_derived\\
  \_\_edit & default or initial values for derived to  fill up form \\
  session & talks to session table in ETS via {\it e\_session} \\
  & which synchronize them \\
  \_\_path & holds URL of current request \\
  \_\_primary\_key & identifies values during update\\
  \_\_cookies & list of tuples {\it \{CookieName, CookieVal\}}\\
  & related to our service\\
  \hline
\end{tabular}


