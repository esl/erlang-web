\subsection{Tuples for HTTP server}
\subsubsection{Overview}Each time our controller end processing some request, it must return a tuple, which tells server what to do. We need some kind of protocol to talk to server to serve proper content or handle HTTP codes (200, 404, 501, etc.). 

In order to tell server what to do, we need to return a situation-specific tuple from each accessible controller method.
\subsubsection{Types of tuples}
\begin{description}
\item[\{redirect, URL\}]- HTTP code will be set to 302, user will be redirected to {\it URL}
\item[\{content, html, Data\}]- {\it Data} is a valid XHTML string, which will be served to the browser. MIME type is set to "text/html"
\item[\{content, text, Data\}]- {\it Data} is a plain text string, which will be served to the browser. MIME type is set to "text/plain"
\item[\{json, Data\}]- {\it Data} will be encoded to JSON format and sent to the browser. MIME type is set to "text/plain"
\item[\{template, Template\}]- selected {\it Template} will be expanded as a response to the request. MIME type is set to "text/html"
\item[template]- when skipping dispatcher, the template specified in URL will be expanded
\item[\{custom, Custom\}]- {\it Custom} is passed to the server. This option is for all the types the framework doesn't handle, but server does
\item[\{error, Code\}]- error with code {\it Code} is generated, template responsible for this kind of error (defined in {\it errors.conf}) is expanded
\item[\{headers, Headers, RetVal\}]- sets the given headers in the session. {\it Headers} is a list of tuples describing which headers should be set. Currently only setting cookies is supported. 
The valid formats are:
\begin{itemize}
\item \{cookies, CookieName, CookieValue\} - sets the cookie with the name {\it CookieName} to the value {\it CookieValue}. The expiration date is set to the end of the session, path is set to {\it /}.
\item \{cookies, CookieName, CookieValue, CookiePath\} - does the same as above, but set path to {\it CookiePath}.
\item \{cookies, CookieName, CookieValue, CookiePath, CookieExpDate\} - does the same as above, but set the expiration date to {\it CookieExpDate}. The date should be in right format: DAY, DD-MMM-YYYY HH:MM:SS GMT
\end{itemize}
{\it RetVal} should be the actual returning value - one of the tuples mentioned above (like \{redirect, URL\} or other).
\end{description} 
