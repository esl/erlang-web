\subsection{Project configuration file}
\subsubsection{Overview}{\it project.conf} file (placed in {\it config} directory) is the place where the settings for application are being held. Having all the things in one configuration file makes it easier to understand, maintain and develop. 

Configuration file contains Erlang tuples: first element of each one is the option name; second - option value. 

All the settings are read during the start of the application, so after every change we must reload them manually, by typing
\begin{verbatim}
e_conf:reinstall().
\end{verbatim}

\subsubsection{Types of options} The following options are being used in the application:
\begin{enumerate}
\item 
\begin{verbatim}
{upload_dir, Dir}
\end{verbatim}

{\bf Dir} specifies the directory, where user uploaded files will be stored. The set directory will be placed inside {\it docroot} folder. By default it is set to {\it "upload"}.

This option could be accessed with command:
\begin{verbatim}
e_conf:upload_dir().
\end{verbatim}

\item
\begin{verbatim}
{default_language, LanguageCode}
\end{verbatim}
{\bf LanguageCode} specifies the default language of translation: if none is set in {\it session:lang} key in {\bf e\_dict}, this one will be used. By default it is set to {\it en}.

This option could be accessed with command: 
\begin{verbatim}
e_conf:default_language().
\end{verbatim}

\item
\begin{verbatim}
{language_files, [LanguageFilesSpecs]}
\end{verbatim}
{\bf LanguageFilesSpecs} specifies the language files with translations. This option has been described deeper in {\it e\_lang} section. 

\item
\begin{verbatim}
{cache_dir, Dir}
\end{verbatim}
{\bf Dir} is the directory, where the cached templates are stored. By default it set to {\it "templates/cache"}.

This option could be accessed with command: 
\begin{verbatim}
e_conf:cache_dir().
\end{verbatim}

\item
\begin{verbatim}
{host, Host}
\end{verbatim}
{\bf Host} is the absolute address of our service. It could be helpful in building absolute links. By default it is set to {\it "localhost"}.

This option could be accessed with command: 
\begin{verbatim}
e_conf:host().
\end{verbatim}

\item
\begin{verbatim}
{primitive_types, ListOfPrimitiveTypes}
\end{verbatim}
{\bf ListOfPrimitiveTypes} defines user-prepared primitive types, which can be used in building application models. We have to provide both {\it wpart\_NameOfTheType} and {\it wtype\_NameOfTheType} modules with corresponding {\bf wpart} and {\bf wtype} behaviours. By default it is set to {\it []}.

This option could be accessed with command: 
\begin{verbatim}
e_conf:primitive_types().
\end{verbatim}

\item
\begin{verbatim}
{debug_mode, Bool}
\end{verbatim}
{\bf Bool} specifies if debugging mode is enabled. If so, all the Erlang errors will be rendered as a error 501 (with explanations) instead of displaying user-specified template. By default it is set to {\it false}.

This option could be accessed with command:
\begin{verbatim}
e_conf:debug_mode().
\end{verbatim}

\item
\begin{verbatim}
{http_port, PortNo}.
{https_port, PortNo}.
\end{verbatim}
{\bf PortNo} specifies the port number for incoming http and https connections. The numbers should be the same as those in server configuration file. They can be used in redirection between protocols running on different than default ports (80 for http and 443 for https).

These options could be accessed with commands:
\begin{verbatim}
e_conf:http_port().
e_conf:https_port().
\end{verbatim}

To easily redirect user from http to https connection, just return:
\begin{verbatim}
{redirect, "https://" ++ e_conf:host() ++ ":" ++ e_conf:https_port() 
            ++ "/" ++ wpart:fget("__path")}.
\end{verbatim}
If server is running on default ports we can return:
\begin{verbatim}
{redirect, "https://" ++ e_conf:host() ++ "/" ++ wpart:fget("__path")}.
\end{verbatim}

\item
\begin{verbatim}
{project_name, Name}.
\end{verbatim}
{\bf Name} specifies the string representing the project name. By default is set to {\it "erlangweb"}.

This option could be accessed with command:
\begin{verbatim}
e_conf:project_name().
\end{verbatim}

\item
\begin{verbatim}
{couchdb_address, URL}.
\end{verbatim}
{\bf URL} specifies the CouchDB address. This address is used only when our DBMS is set to CouchDB for communicating with CouchDB server. By default is set to {\it "http://localhost:5984/"}.

This option could be accessed with command:
\begin{verbatim}
e_conf:couchdb_address().
\end{verbatim}

\item
\begin{verbatim}
{dbms, DBMS}.
\end{verbatim}
{\bf DBMS} is the type of the DB engine used in our project. It could either {\it mnesia} or {\it couchdb}. This option has been described deeper in e\_db section. By default it is set to {\it mnesia}.

\end{enumerate}

We can also place our own options inside the {\it project.conf} file. These settings could be found in {\it e\_conf} ets table under our own defined key. 

\subsubsection{Example} This is the simple example of {\it project.conf} file
\begin{Verbatim}[numbers=left]
{upload_dir, "user_upload"}.
{host, "example.org"}.
{default_language, de}.
{language_files, [{en, "config/languages/en.conf"}, 
                  {de, "config/languages/de.conf"}]}.
{admin_logins, [adam, michael]}.
{primitive_types, [embedded_video]}.
\end{Verbatim}


The configuration of our application will be as follows:
\begin{itemize}
\item upload directory will be set to {\it "user\_upload"}
\item hostname will be set to {\it "example.org"}
\item default language of translation will be {\it de}
\item there will be two translation files: for {\it en} ({\it "config/languages/en.conf"}) and for {\it de} ({\it "config/languages/de.conf"})
\item cache directory will be set to {\it "templates/cache"} (by default)
\item user-defined setting, {\bf admin\_logins} will be set to {\it $\left[ adam, michael \right]$}
\item there will be new primitive type: {\it embedded\_video}
\end{itemize}
